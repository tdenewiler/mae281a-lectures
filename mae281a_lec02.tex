\mainmatter%
\setcounter{page}{1}

\lectureseries[\course]{\course}

\auth[\lecAuth]{Lecturer: \lecAuth\\ Scribe: \scribe}
\date{January 7, 2010}

\setaddress%

% the following hack starts the lecture numbering at 1
\setcounter{lecture}{1}
\setcounter{chapter}{1}

\lecture{Limit Cycles}%
\label{lec:mae281a_lec02}

\section{Limit Cycles Recap}
Last lecture we studied linear and nonlinear limit cycles.
An example of a linear limit cycle is a  harmonic oscillator given by

\begin{equation*}
\ddot{x} + x = 0
\end{equation*}

and the state space is shown in Figure~\ref{fig:02secondorder}.

\begin{figure}[ht!]
\centering
\includegraphics[width=.4\textwidth]{images/01secondorder}%
\label{fig:02secondorder}
\caption{Limit cycle for second order harmonic oscillator.}
\end{figure}

An example of a nonlinear limit cycle is a Van der Pol oscillator given by

\begin{equation*}
\ddot{x} + (x^2-1)\ddot{x} + x = 0
\end{equation*}

and the state space is shown in Figure~\ref{fig:02vdplc}.
Note that there is positive damping for $|x|>1$ and negative damping for $|x|<1$.

\begin{figure}[ht!]
\centering
\includegraphics[width=.4\textwidth]{images/02vdplc}
\caption{Limit cycle for Van der Pol oscillator.}%
\label{fig:02vdplc}
\end{figure}

The harmonic oscillator has a period of $2\pi$ and the Van der Pol oscillator has a base harmonic accompanied by lower frequencies and can be modeled using a Fourier series.

\subsection{Isolated Periodic Orbit}
An isolated periodic orbit will have a single amplitude.
Examples of Van der Pol oscillators are
\begin{itemize}
\item The human heart.
\item An LC circuit with a tunnel diode.
\item A compressor in ``surge'' mode where there is a pumping motion in flow and pressure as found in jet engines.
\end{itemize}

\section{Chaos}
Chaotic systems are ``random'' yet created by a deterministic system like $\dot{x}=f(x)$.
It is interesting to look at the lowest order autonomous systems that can generate chaotic behavior which have been found to be of order $n=3$ in continuous time and order $n=1$ in discrete time.
In this course we will be studying continuous time for the most part as it is more relevant to mechanical and aerospace systems than discrete time is.

\subsection{Lorenz Equations}
The Lorenz equations given in (\ref{eq:02lorenz}) are an example of chaotic dynamics in continuous time.
The equations are not periodic and cannot be represented by a Fourier series.
\begin{align}
\label{eq:02lorenz}
\dot{x} &= \sigma(y - x) \nonumber \\
\dot{y} &= \rho x - y - xz \\
\dot{z} &= -\beta z + xy \nonumber
\end{align}
where $(x,y,z)\in\mathbb{R}^3$ and $\sigma, \rho, \beta > 0$.

To find the equilibria of the Lorenz equations we set $\dot{x}=f(x)=0$ and solve the set of ODEs in (\ref{eq:02lorenz}) to find
\begin{align*}
\dot{x} &= \sigma(y-x) = 0 \Rightarrow y = x \\
\dot{z} &= -\beta z + xy = 0 \Rightarrow z = \frac{y^2}{\beta} \\
\dot{y} &= \rho x-y-xz = 0 \Rightarrow (\rho-1)y - y\frac{y^2}{\beta} = 0 \Rightarrow y = 0, \pm\sqrt{\beta(\rho-1)}
\end{align*}
which yields the equilibria as
\begin{align*}
(x,y,z) = \begin{cases} (0,0,0) \\ (\sqrt{\beta(\rho-1)}, \sqrt{\beta(\rho-1)}, \rho-1) \\ (-\sqrt{\beta(\rho-1)}, -\sqrt{\beta(\rho-1)}, \rho-1) \end{cases}
\end{align*}

\section{Qualitative Behavior Near Equilibria}
This section corresponds to \S2.3 of Khalil.
Let

\begin{equation*}
\dot{x} = f(x), \qquad f(p) = 0
\end{equation*}

so that the point $p$ is an equilibrium.
We can use a Taylor series expansion about $x=p$ to find
\begin{align*}
\dot{x} = \underbrace{f(p)}_{=0} + \underbrace{\left.\frac{\partial f(x)}{\partial x}\right|_{x=p}}_{\text{constant square matrix}} \underbrace{(x-p)}_{y} + \text{H.O.T.}
\end{align*}
Let

\begin{equation*}
A \triangleq \left.\frac{\partial f(x)}{\partial x}\right|_{x=p}
\end{equation*}

and then the linearization is $\dot{y}=Ay$. Note that
\begin{align*}
\frac{\partial f(x)}{\partial x} = \left[\begin{array}{c c c} \frac{\partial f(x_1)}{\partial x_1} & \cdots & \frac{\partial f(x_1)}{\partial x_n} \\ \vdots & \ddots & \vdots \\ \frac{\partial f(x_n)}{\partial x_1} & \cdots & \frac{\partial f(x_n)}{\partial x_n} \end{array}\right]
\end{align*}

For second order systems the state space is the plane spanned by $x_1 = x$ and $x_2 = \dot{x}$ as we have seen in several previous examples.

\subsection{Qualitative Behavior of Second-Order Linear Systems}
For the system

\begin{equation*}
\dot{x} = Ax
\end{equation*}

we have to look at the eigenvalues and eigenvectors of $A$.
There are three possible Jordan forms of $A$ given as
\begin{align*}
\left[\begin{array}{c c} \lambda_1 & 0 \\ 0 & \lambda_2 \end{array}\right], \qquad
\left[\begin{array}{c c} \alpha & -\beta \\ \beta & \alpha \end{array}\right], \qquad
\left[\begin{array}{c c} \lambda & k \\ 0 & \lambda \end{array}\right]
\end{align*}

In the first case we have $\lambda_1\neq\lambda_2\neq0$.
Figure~\ref{fig:02stable1} shows the stable mode for $\lambda_1, \lambda_2<0$, Figure~\ref{fig:02unstable1} shows the unstable mode for $\lambda_1, \lambda_2>0$ and Figure~\ref{fig:02saddle1} shows the saddle point for $\lambda_1>0$ and $\lambda_2<0$.

\begin{figure}[ht!]
\centering
\subfloat[$\lambda_1,\lambda_2<0$, stable mode.]{%
\label{fig:02stable1}\includegraphics[width=0.35\textwidth]{images/02stable1}
} \hfill
\subfloat[$\lambda_1,\lambda_2>0$, unstable mode.]{%
\label{fig:02unstable1}\includegraphics[width=0.35\textwidth]{images/02unstable1}
} \\
\subfloat[$\lambda_1>0,\lambda_2<0$, saddle point.]{%
\label{fig:02saddle1}\includegraphics[width=0.35\textwidth]{images/02saddle1}
}
\caption{Case 1 with $\lambda_1\neq\lambda_2\neq0$.}%
\label{fig:02case1}
\end{figure}

In the second case we have $\lambda_{1,2}=\alpha\pm\ j\beta$.
Figure~\ref{fig:02stable2} shows the stable mode for $\alpha<0$, Figure~\ref{fig:02unstable2} shows the unstable mode for $\alpha>0$ and Figure~\ref{fig:02center2} shows the oscillatory behavior that occurs when $\alpha=0$.

\begin{figure}[ht!]
\centering
\subfloat[$\alpha<0$, stable mode.]{%
\label{fig:02stable2}\includegraphics[width=0.35\textwidth]{images/02stable2}
} \hfill
\subfloat[$\alpha>0$, unstable mode.]{%
\label{fig:02unstable2}\includegraphics[width=0.35\textwidth]{images/02unstable2}
} \\
\subfloat[$\alpha=0$, center point.]{%
\label{fig:02center2}\includegraphics[width=0.35\textwidth]{images/02center2}
}
\caption{Case 2 with $\lambda_{1,2} = \alpha\pm j\beta$.}%
\label{fig:02case2}
\end{figure}

In the third case we have $\lambda_1=\lambda_2=\lambda\neq0$ and we can have $k=0$ or $k=1$.
If $k=0$ then there are two independent eigenvectors and Figure~\ref{fig:02lltwo3} shows the stable mode for $\lambda<0$ and Figure~\ref{fig:02lgtwo3} shows the unstable mode for $\lambda>0$.
When $k=1$ there is only a single independent eigenvector and troublesome behavior exists.
Figure~\ref{fig:02llone3} shows what happens when $\lambda<0$ and Figure~\ref{fig:02lgone3} shows what happens when $\lambda>0$.

\begin{figure}[ht!]
\centering
\subfloat[$\lambda<0$, two independent eigenvectors.]{%
\label{fig:02lltwo3}\includegraphics[width=0.35\textwidth]{images/02lltwo3}
} \hfill
\subfloat[$\lambda>0$, two independent eigenvectors.]{%
\label{fig:02lgtwo3}\includegraphics[width=0.35\textwidth]{images/02lgtwo3}
} \\
\subfloat[$\lambda<0$, one independent eigenvector.]{%
\label{fig:02llone3}\includegraphics[width=0.35\textwidth]{images/02llone3}
} \hfill
\subfloat[$\lambda>0$, one independent eigenvector.]{%
\label{fig:02lgone3}\includegraphics[width=0.35\textwidth]{images/02lgone3}
}
\caption{Case 3 with $\lambda_1=\lambda_2=\lambda\neq0$.}%
\label{fig:02case3}
\end{figure}
